\documentclass{article}
\usepackage[utf8]{inputenc}

\usepackage{graphicx}
\graphicspath{ {images/} }

%For headers & footers
\usepackage{fancyhdr}
\pagestyle{fancy}
\lhead{\includegraphics[scale=0.2]{Logo}}
\chead{Kartrobot}
\rhead{2016-mm-dd}

\lfoot{Konstruktion med mikrodatorer}
\rfoot{Grupp 3 \\ ev. e-post till projektgrupp}

\renewcommand{\headrulewidth}{0.4pt}
\renewcommand{\footrulewidth}{0.4pt}


\title{Systemskiss}
\author{Patrik Sletmo}
\date{September 2016}


\begin{document}

\thispagestyle{empty}

{
\sffamily
\centering
\large


{\huge 
Systemskiss
}

{\large
Patrik Sletmo
}

{\large
Version x.y
}

\vspace{3.5cm}

Status
\begin{center}
\begin{tabular}{ | c | c | c | } 
\hline
Granskad & snubbe & 2016-mm-dd \\
\hline
Godkänd & snubba & 2016-mm-dd\\
\hline
\end{tabular}
\end{center}
}

\clearpage

{
\sffamily
\centering
\large


{\huge 
Projektidentitet
}

{\large
Projektgruppsnummer, årtal/termin, projektgruppsnamn \\ Linköpings tekniska högskola, institution 
}

\vspace{3.5cm}

Status
\begin{center}
\begin{tabular}{ | c | c | c | c |} 
\hline
Namn & Ansvar & Telefon & E-post \\  
\hline
Ett namn & ett ansvar & ett telefon & ett e-post \\
\hline
\end{tabular}
\end{center}
}

\begin{center}
\textbf{Hemsida}: https://github.com/SebastianCallh/kartoffel-tsea29
\end{center}

\begin{center}
\textbf{Kund}: Kundbeskrivning, 581 00 LINKÖPING, \\
kundtelefon: 013-11 00 00, fax: 013-10 19 02, e-postadress \\
\textbf{Kontaktperson hos kund}: namn, tel., mobil-nr., e-postadress 
\end{center}

\begin{center}
\textbf{Kursansvarig}: namn, kontorsrum, tel., e-postadress \\
\textbf{Handledare}: namn, tel., mobil-nr., e-postadress 
\end{center}
\clearpage



\renewcommand*\contentsname{Innehållsförteckning}
\tableofcontents
\clearpage


{
\sffamily
\centering
\large


{\huge 
Dokumenthistorik
}
Status
\begin{center}
\begin{tabular}{ | c | c | c | c | c |} 
\hline
\textbf{Version} & \textbf{Datum} & \textbf{Utförda ändringar} & \textbf{Utförd av } & \textbf{Granskad} \\  
\hline
0.1 & 2016-09-xx & Första utkastet &  john doe & icke \\
\hline
\end{tabular}
\end{center}
}

\clearpage


\section{Inledning}
Den här systemskissen beskriver i mer teknisk detalj konstruktionen av roboten och tillhörande mjukvara. Dokumentet ska reflektera vår faktiska relisation av projektet och kan mycket väl komma att ändras allt eftersom projektet fortskrider och designbesluts behöver omvärderas.

\section{Systemöversikt}
Övergripande text om moduler, bussar, dylikt
Blockschema över roboten

\section{Huvudenhet}
Text och blockschema
\subsection{Kommunikationsenhet}
Flödesdiagram
\subsection{Logikenhet}
Flödesdiagram
Bild på navigation
Bild på hur kartan sparas
Bild på navigering

\section{Styrenhet}
Text och blockschema

\section{Sensorenhet}
Text och blockschema
Bild på laser

\section{Mjukvaruklient}
Text och skiss på GUI

\end{document}
