\documentclass{article}
\usepackage[utf8]{inputenc}

%For headers & footers
\usepackage{fancyhdr}
\pagestyle{fancy}
\lhead{LOGO}
\chead{Kartrobot}
\rhead{2016-mm-dd}

\lfoot{Konstruktion med mikrodatorer}
\rfoot{Grupp 3 \\ ev. e-post till projektgrupp}

\renewcommand{\headrulewidth}{0.4pt}
\renewcommand{\footrulewidth}{0.4pt}


\title{Systemskiss}
\author{Patrik Sletmo}
\date{September 2016}


\begin{document}

\thispagestyle{empty}

{
\sffamily
\centering
\large


{\huge 
Systemskiss
}

{\large
Patrik Sletmo
}

{\large
Version x.y
}

\vspace{3.5cm}

Status
\begin{center}
\begin{tabular}{ | c | c | c | } 
\hline
Granskad & snubbe & 2016-mm-dd \\
\hline
Godkänd & snubba & 2016-mm-dd\\
\hline
\end{tabular}
\end{center}
}

\clearpage

{
\sffamily
\centering
\large


{\huge 
Projektidentitet
}

{\large
Projektgruppsnummer, årtal/termin, projektgruppsnamn \\ Linköpings tekniska högskola, institution 
}

\vspace{3.5cm}

Status
\begin{center}
\begin{tabular}{ | c | c | c | c |} 
\hline
Namn & Ansvar & Telefon & E-post \\  
\hline
Ett namn & ett ansvar & ett telefon & ett e-post \\
\hline
\end{tabular}
\end{center}
}

\begin{center}
\textbf{Hemsida}: https://github.com/SebastianCallh/kartoffel-tsea29
\end{center}

\begin{center}
\textbf{Kund}: Kundbeskrivning, 581 00 LINKÖPING, \\
kundtelefon: 013-11 00 00, fax: 013-10 19 02, e-postadress \\
\textbf{Kontaktperson hos kund}: namn, tel., mobil-nr., e-postadress 
\end{center}

\begin{center}
\textbf{Kursansvarig}: namn, kontorsrum, tel., e-postadress \\
\textbf{Handledare}: namn, tel., mobil-nr., e-postadress 
\end{center}
\clearpage



\renewcommand*\contentsname{Innehållsförteckning}
\tableofcontents
\clearpage


{
\sffamily
\centering
\large


{\huge 
Dokumenthistorik
}
Status
\begin{center}
\begin{tabular}{ | c | c | c | c | c |} 
\hline
\textbf{Version} & \textbf{Datum} & \textbf{Utförda ändringar} & \textbf{Utförd av } & \textbf{Granskad} \\  
\hline
0.1 & 2016-09-xx & Första utkastet &  john doe & icke \\
\hline
\end{tabular}
\end{center}
}

\clearpage


\section{Inledning}
Den här systemskissen beskriver i mer teknisk detalj konstruktionen av roboten och tillhörande mjukvara. Dokumentet ska reflektera vår faktiska relisation av projektet och kan mycket väl komma att ändras allt eftersom projektet fortskrider och designbesluts behöver omvärderas.

\section{Systemöversikt}
Övergripande text om moduler, bussar, dylikt
Blockschema över roboten

\section{Huvudenhet}
Text och blockschema
\subsection{Kommunikationsenhet}
Flödesdiagram
\subsection{Logikenhet}
Flödesdiagram
Bild på navigation
Bild på hur kartan sparas
Bild på navigering

\section{Styrenhet}
Text och blockschema

\section{Sensorenhet}
Text och blockschema

\subsection{Avståndssensor}
För att med så hög precision som möjligt kunna mäta avståndet till de kringläggande väggarna använder sig roboten av en LIDAR-Lite v2 laseravståndsmätare. Denna sensor är monterad tillsammans med ett servo på toppen av roboten för att kunna snurra runt och mäta avstånd i 360°. Den sensordata som rapporteras av lasern kombineras med data från ett gyro som också monterats på rotationsservot för att kunna ge en beskrivning av området runt roboten.

\subsubsection{Begränsningar}
Att montera två sensorer på en roterande del medför en del problem med hur t.ex. sladdar hanteras. Servot måste rotera i svepande rörelser fram och tillbaka för att inte dra sönder sladdarna och längden på sladdarna får inte vara för lång för att undvika att de hamnar ivägen.

\subsubsection{Rotation}
För att både kunna leverera korrekt data till huvudenheten samt undvika att förstöra hårdvaran på grund av utdragna sladdar behöver detektionen av sensorservots nuvarande rotation hålla sig innanför en viss felmarginal. Dels så rapporterar servot AX-12 sin nuvarande rotationsvinkel via dess dataport och dels kommer gyrosensorn ge en uppskattning om hur servot roterat. Blir felet för stort kan AX-12-servots rotation manuellt ändras till en given vinkel. I och med att data om rotationen kommer rapporteras från både sensorenheten (gyro) och styrenheten (servo) så sker beräkningen av vinkeln i huvudenheten.

\section{Mjukvaruklient}
Text och skiss på GUI



\end{document}
