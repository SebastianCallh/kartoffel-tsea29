\documentclass{article}
\usepackage[utf8]{inputenc}

%For headers & footers
\usepackage{fancyhdr}
\pagestyle{fancy}
\lhead{LOGO}
\chead{Kartrobot}
\rhead{2016-mm-dd}

\lfoot{Konstruktion med mikrodatorer}
\rfoot{Grupp 3 \\ ev. e-post till projektgrupp}

\renewcommand{\headrulewidth}{0.4pt}
\renewcommand{\footrulewidth}{0.4pt}


\title{Systemskiss}
\author{Patrik Sletmo}
\date{September 2016}


\begin{document}

\thispagestyle{empty}

{
\sffamily
\centering
\large


{\huge 
Systemskiss
}

{\large
Patrik Sletmo
}

{\large
Version x.y
}

\vspace{3.5cm}

Status
\begin{center}
\begin{tabular}{ | c | c | c | } 
\hline
Granskad & snubbe & 2016-mm-dd \\
\hline
Godkänd & snubba & 2016-mm-dd\\
\hline
\end{tabular}
\end{center}
}

\clearpage

{
\sffamily
\centering
\large


{\huge 
Projektidentitet
}

{\large
Projektgruppsnummer, årtal/termin, projektgruppsnamn \\ Linköpings tekniska högskola, institution 
}

\vspace{3.5cm}

Status
\begin{center}
\begin{tabular}{ | c | c | c | c |} 
\hline
Namn & Ansvar & Telefon & E-post \\  
\hline
Ett namn & ett ansvar & ett telefon & ett e-post \\
\hline
\end{tabular}
\end{center}
}

\begin{center}
\textbf{Hemsida}: https://github.com/SebastianCallh/kartoffel-tsea29
\end{center}

\begin{center}
\textbf{Kund}: Kundbeskrivning, 581 00 LINKÖPING, \\
kundtelefon: 013-11 00 00, fax: 013-10 19 02, e-postadress \\
\textbf{Kontaktperson hos kund}: namn, tel., mobil-nr., e-postadress 
\end{center}

\begin{center}
\textbf{Kursansvarig}: namn, kontorsrum, tel., e-postadress \\
\textbf{Handledare}: namn, tel., mobil-nr., e-postadress 
\end{center}
\clearpage



\renewcommand*\contentsname{Innehållsförteckning}
\tableofcontents
\clearpage


{
\sffamily
\centering
\large


{\huge 
Dokumenthistorik
}
Status
\begin{center}
\begin{tabular}{ | c | c | c | c | c |} 
\hline
\textbf{Version} & \textbf{Datum} & \textbf{Utförda ändringar} & \textbf{Utförd av } & \textbf{Granskad} \\  
\hline
0.1 & 2016-09-xx & Första utkastet &  john doe & icke \\
\hline
\end{tabular}
\end{center}
}

\clearpage


\section{Inledning}
Den här systemskissen beskriver i mer teknisk detalj konstruktionen av roboten och tillhörande mjukvara. Dokumentet ska reflektera vår faktiska relisation av projektet och kan mycket väl komma att ändras allt eftersom projektet fortskrider och designbesluts behöver omvärderas.

\section{Systemöversikt}
Övergripande text om moduler, bussar, dylikt
Blockschema över roboten

\section{Huvudenhet}
\includegraphics{Huvudmodul_oversikt_blockschema}
\caption{Figur X - Översiktsblockschema över huvudenheten}
Figur X visar hur kommunikationen sker inom huvudenheten på ett översiktligt plan. Processorn innehåller minne där vi kommer lagra data, variabler, konstanter samt själva programmet som styr huvudenhetens logik (se \textit{Logikenhet}). Processorn kommunicerar till omvärlden på två sätt, via en Bluetoothenhet och via en I2C-buss. Kommunikationen via Bluetooth bygger på en slags handskakningsprincip mellan processorn och den PC som Bluetoothenheten är kopplad till. Processorn skickar signalen \textit{Clear to send} till PC för att tala om att den är redo att ta emot data. PC:n skickar signalen \textit{Request to send} till processorn när den är redo att ta emot data. I2C-bussen används för kommunikation mellan huvud-, sensor- och styrenhet. Huvudenheten är ensam master på bussen och styr på så vis kommunikaitonen. Följdaktligen är den även ansvarig för att generera klockpulsen SCL. När antingen sensor- eller styrenheten är redo att skicka data eller låta huvudenheten läsa data, kommer de skicka ett interrupt till huvudenhetens processor. Den processor vi har tänkt att använda oss utav i huvuenhten är en ATmega1284 processor, med 128kB flashminne samt extra 20 kB för variabler och konstanter och en klocka på 0-20 MHz. 

\subsection{Kommunikationsenhet}
Flödesdiagram
\subsection{Logikenhet}
Flödesdiagram
Bild på navigation
Bild på hur kartan sparas
Bild på navigering

\section{Styrenhet}
Text och blockschema

\section{Sensorenhet}
Text och blockschema
Bild på laser

\section{Mjukvaruklient}
Text och skiss på GUI



\end{document}
