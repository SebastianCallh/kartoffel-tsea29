\documentclass{article}
\usepackage[utf8]{inputenc}
\usepackage[english, swedish]{babel}
\usepackage{graphicx}
\usepackage{float}

%For headers & footers
\usepackage{fancyhdr}
\pagestyle{fancy}
\lhead{\includegraphics[scale=0.1516]{Logo}}
\chead{Kartrobot}
\rhead{2016-mm-dd}

\lfoot{Konstruktion med mikrodatorer}
\rfoot{Grupp 3 \\ ev. e-post till projektgrupp}

\renewcommand{\headrulewidth}{0.4pt}
\renewcommand{\footrulewidth}{0.4pt}


\title{Projektplan}
\author{Patrik Sletmo}
\date{September 2016}


\begin{document}
\graphicspath{{./images/}}
\selectlanguage{swedish}

\thispagestyle{empty}

{
\sffamily
\centering
\large


{\huge
Projektplan
}

{\large
Patrik Sletmo
}

{\large
Version 0.1
}

\vspace{13.5cm}

Status
\begin{center}
\begin{tabular}{ | c | c | c | }
\hline
Granskad & Patrik Sletmo & 2016-09-22 \\
\hline
\end{tabular}
\end{center}
}

\clearpage

\vspace*{\fill}
{
\sffamily
\centering
\large


{\huge
Projektidentitet
}

{\large
Grupp 3, 16/HT, KarToffel \\ Linköpings tekniska högskola, institution
}

\vspace{0.5cm}

\begin{table}[H]
\centering
\begin{tabular}{ | c | c | c | c |}
\hline
Namn & Ansvar & Telefon & E-post \\
\hline
Patrik Sletmo & Projektledare & 070 783 57 61 & patsl736@student.liu.se \\
\hline
Rebecca Lindblom &  & 073 436 40 79 & rebli156@student.liu.se \\
\hline
Matildha Sjöstedt &  & 070 515 84 11 & matsj696@student.liu.se \\
\hline
Sebastian Callh &  & 073 820 46 64 & sebca553@student.liu.se \\
\hline
Anton Dalgren &  & 076 836 51 56 & antda685@student.liu.se \\
\hline
Matilda Dahlström &  & 070 636 33 52 & matda715@student.liu.se \\
\hline
\end{tabular}
\end{table}
}

\begin{center}
\textbf{Hemsida}: https://github.com/SebastianCallh/kartoffel-tsea29
\end{center}

\begin{center}
\textbf{Kund}: Mattias Krysander, 013 - 28 2198 , matkr@isy.liu.se
\end{center}

\begin{center}
\textbf{Kursansvarig}: Tomas Svensson, 3B 528, +46 (0)13 28 1368, tomas.svensson@liu.se \\
\textbf{Handledare}: Anders Nilsson, 3B 512, +46 (0)13 28 2635, anders.p.nilsson@liu.se
\end{center}
\vspace*{\fill}
\clearpage



\renewcommand*\contentsname{Innehållsförteckning}
\tableofcontents
\clearpage

\section{Beställare}
Beställare och köpare av projektet är Mattias Krysander, 013 - 28 2198, matkr@isy.liu.se

\section{Översiktlig beskrivning av projektet}

\subsection{Syfte och mål}
Syftet med det här projektet är att få förståelse för mikroprocessorer och kommunikationen mellan
dessa. Målet med detta projekt är att producera en robot som autonomt läser av ett rum och i realtid  ritar upp en karta över rummet.

\subsection{Leveranser}
Alla leveranser ska ske genom mail till beställaren i PDF-format, såvida inget annat står angivet.

\begin{itemize}
    \item Kravspecifikationen ska vara godkänd senast den 2016-09-13.
    \item Första versionen av projektplan, tidsplan och systemskiss ska vara inlämnad till beställaren senast 2016-09-22.
    \item Slutgiltiga versionen av projektplan, tidsplan och systemskiss ska vara inlämnad till beställaren senast 2016-09-29.
    \item Första versionen av designspecifikationen ska vara inlämnad till beställaren senast 2016-11-01.
    \item Slutgiltiga versionen av designspecifikationen ska vara godkänd senast 2016-11-04.
    \item Efterstudien ska vara inlämnad senast 2016-12-21.
    \item Utrustningen ska vara inlämnad senast 2016-12-22.
    \item Roboten ska levereras och redovisas senast vecka 51 till beställaren personligen.
    \item Teknisk dokumentation ska vara inlämnad senast tre arbetsdagar före redovisning.
    \item Användarhandledning ska vara inlämnad senast tre arbetsdagar före redovisning.
\end{itemize}


\subsection{Begränsningar}
Roboten ska ej förväntas fungera i en miljö utan tydliga konturer och hörn eller där det finns föremål som rör sig. Roboten förväntas heller inte fungera optimalt i ojämn terräng. Under projektet kommer ej någon omfattande utvärdering göras vad gäller vilka programmeringsspråk eller komponenter som är mest optimala.

\section{Fasplan}

\subsection{Före projektstart}
Innan projektets början ska en kravspecifikation tillsammans med projektplan, systemskiss och tidsplan färdigställas.

\subsubsection{Bilda projektgrupp}
En projektgrupp på sex studenter som läser kursen Konstruktion med mikrodatorer (TSEA29) ska bildas och rapporteras till kursens examinator. Samtidigt som gruppen bildas ska en projektledare utses.

\subsubsection{Kravspecifikation}
En kravspecifikation enligt LIPS-modellen ska skrivas och godkännas av projektets beställare, se leveranser.

\subsubsection{Ansvarsfördelning}
Endast rollen ``Projektledare'' bestäms, som även får rollen dokumentansvarig. Resten av projektmedlemmarna får rollen utvecklare. 

\subsubsection{Tidsplan}
En tidsplan där alla projektmedlemmars tider planeras ska genomföras och godkännas av projektets beställare.

\subsubsection{Systemskiss}
En översiktlig beskrivning av systemet där det framgår hur produkten ska konstrueras ska skapas och godkännas av projektets beställare. Från skissen ska det gå att avgöra vilka moduler systemet innehåller och dokumentet blir ett underlag för då konstruktionen ska delas in i arbetsblock.

\subsection{Under projektet}
Under projektet ska produkten designas, utvecklas och testas. Gruppen ska också tid- och statusrapportera.

\subsubsection{Design}
En designspecifikation ska tas fram och godkännas av handledaren och sedan lämnas till beställaren, se leveranser.

\subsubsection{Utveckling}
Produktens alla delar ska konstrueras. Fysiska delar ska monteras ihop och mjukvara ska utvecklas.

\subsubsection{Tester}
Tester ska definieras och utföras på olika delar och nivåer av produkten för att säkerställa att produkten uppfyller kraven formulerade i kravspecifikationen.

\subsubsection{Planering}
Möten ska hållas och planering ska ske. Om så behövs redigeras en tidigare planering för att fungera under nya förutsättningar. Beslutsmöten ska hållas mellan projektgruppen och beställaren.

\subsubsection{Rapportering}
Tidrapportering ska skickas till beställaren veckovis under projektet. På beställarens begäran ska statusrapport skickas in.

\subsection{Efter projektet}
Efter projektet ska roboten presenteras inför beställare och kund, samt ingå i en tävling. En efterstudie ska också genomföras med syfte att utvärdera arbetet med projektet. Efter tävlingen ska roboten demonteras och alla komponenter återlämnas. Därmed finns ej någon avsikt att underhålla eller vidareutveckla roboten eller någon av delprodukterna. Projektgruppen kommer upplösas efter projektets slut.

\section{Organisationsplan för hela projektet}
Detta kapitel handlar om strukturen som används inom projektet både hos kunden och projektgruppen. Mer specifikt går den in på hur projektgruppen ska samarbeta, vad som gäller för samarbetet och arbetsinnehållet för de olika ansvarsposterna.

\subsection{Organisationsplan per fas}
\begin{figure}
\includegraphics{Organisationsplan}
\caption{Organisation i projektet}
\label{fig:organisationsplan}
\end{figure}
Gruppen ska ha samma organisation under hela projektet. Organisationen visas i Figur ~\ref{fig:organisationsplan}.

\subsection{Villkor för samarbetet inom projektgruppen}
\label{subsec:villkorsamarbete}

I slutet av varje vecka ska nästkommande veckas schema fastställas tillsammans med hela gruppen. Ifall man inte kan närvara på alla tillfällen som bestäms är det okej så länge man meddelar det i förväg. Även plötsliga orsaker till frånvaro är okej vid giltig frånvaro (t.ex. sjukdom, personliga problem, etc.).
\newline\newline
Till varje tillfälle som gruppen arbetar ska samtliga medlemmar komma väl förberedda. Har gruppen bestämt något som ska vara gjort till nästa gång ska varje gruppmedlem tagit på sig ansvaret att få det gjort i den mån det är möjligt.
\newline\newline
I början av varje vecka ska ett utvärderande möte hållas. Under det utvärderande mötet kan gruppens medlemmar lyfta fram åsikter om vad som fungerat bra respektive dåligt med projektet och samarbetet den senaste veckan. Det är okej att lyfta fram både positiv och negativ kritik så länge den framförs på ett professionellt sätt och kan argumenteras för. Kritik som uppkommer under veckans gång ska inte framföras förrän det utvärderande mötet om möjligt.
\newline\newline
Under arbetets gång ska arbetsuppgifterna fördelas likvärdigt i den mån det är möjligt. Eftersom det enligt beställarens direktiv ska avsättas exakt 160 timmar per person till projektet efter beslutspunkt 2 kommer alla medlemmar arbeta lika länge på projektet, även om det inte är synonymt med lika mycket. Gruppmedlemmar som inte har lagt sina 160 timmar riskerar att inte bli godkända enligt kursens upplägg.
\newline\newline
Alla gruppmedlemmar ska göra sitt bästa utan att överanstränga sig för att bidra till en kvalitativ leverans.
\newline\newline
Gruppen ska främst fokusera på fakta när beslut i gruppen tas. I de fall där någon gruppmedlem har tidigare erfarenhet av att ett annat alternativ fungerat bättre än det som stöds av fakta ska det övervägas och väljas ifall gruppmedlemmen kan framföra övertygande argument för sitt alternativ. Känslor ska inte tas hänsyn till för andra beslut än tidsplanering.

\subsection{Definition av arbetsinnehåll och ansvar}
\subsubsection{Projektledare}
Projektledaren ansvarar för att leda arbetet i gruppen framåt och är ansvarig utgivare för alla dokument gruppen producerar. Det är projektledarens uppgift att se till så att alla gruppmedlemmar arbetar utefter de riktlinjer som ställts i projektplanen. Projektledaren ansvarar också för att all kravställd dokumentation levereras till beställaren och att tidsrapportering för hela gruppen skickas in vid i förhand bestämda tillfällen. Utöver projektledaransvaret är projektledaren även utvecklare till 90\%.

\subsubsection{Utvecklare}
Varje modul tilldelas ett team av utvecklare, som tillsammans ansvarar för att utveckla och skriva tester för deras modul. En utvecklare kan ha olika underansvar i sitt team, t.ex. ansvar för designen, tester eller för en specifik detalj i en modul beroende på hennes erfarenheter och teamets behov.

\section{Dokumentplan}

Alla dokument som ingår i projektet listas i tabellen nedan. Alla dokument skrivs på svenska. Alla medlemar i projektgruppen har behörighet att läsa samt redigera alla dokument som skapas av projektgruppen. Beställaren och kunden har endast behörighet att läsa de dokument som distribueras till dem. Dokumenten versionshanteras med hjälp av versionshanteringssystemet git. När ett dokument som inte är internt för projektgruppen anses färdigt att distribueras skickas det till den berörda parten. Vid distribution anges ett versionsnummer till dokumentet på formen X.Y där varje första distribution börjar med versionsnummer 0.1, och varje mindre revidering ökar versionsnummret på höger sida om punkten. Vid en större revision ökar siffran på vänster sida om punkten, och då nollställs även siffran till höger. Versionsnummret ändras endast vid granskning av dokumentet av utomstående part.

\begin{table}[H]
\centering
\resizebox{\textwidth}{!}{\begin{tabular}{ | l | l | l | l | l | l | }
% TODO Fix line break in table
\hline
\textbf{Dokument} & \textbf{Ansvarig} & \textbf{Godkänns av} & \textbf{Syfte} & \textbf{Distribueras till} & \textbf{Färdigdatum} \\
\hline
Kravspecifikation & Patrik & Mattias & Definierar alla krav på systemet & Mattias och gruppen & 2016-mm-dd\\
\hline
Projektplan & Patrik & Mattias & Hjälpmedel för hur & Mattias och gruppen & 2016-mm-dd \\
& & & projektet ska genomföras & & \\
\hline
Tidsplan & Patrik & Mattias & Hjälpmedel för att hålla ekonomikrav & Mattias och gruppen & 2016-mm-dd\\
\hline
Systemskiss & Patrik & Mattias & Underlag för designspecifikation & Mattias och gruppen & 2016-mm-dd\\
\hline
Designspecifikation & Patrik & Anders & Underlag för konstruktionsarbetet & Mattias och gruppen & 2016-mm-dd\\
\hline
Tidrapporter & Patrik & Mattias & Visar tidsfördelningen inom gruppen & Mattias och gruppen & 2016-mm-dd\\
\hline
Uppdaterad tidsplan & Patrik & Mattias & Visar om projektet håller tidsplanen & Mattias och gruppen & 2016-mm-dd\\
\hline
Teknisk dokumentation & Patrik & Mattias & Beskrivning av de tekniska lösningarna & Mattias och gruppen & 2016-mm-dd\\
\hline
Användarhandledning & Patrik & Mattias & Manual för den tänkta användaren & Mattias och gruppen & Tre dagar före redovisning\\
\hline
Efterstudie & Patrik & - & Reflektera över förbättrningar inför & Mattias och gruppen & 2016-mm-dd \\
& & & framtida projekt & & \\
\hline
Statusrapport & Patrik & Mattias & Visar om projektet håller tidsplanen & Mattias och gruppen & Vid begäran\\
\hline
\end{tabular}}
\end{table}

\section{Utvecklingsmetodik}
Arbetet kommer mestadels ske moduluppdelat där grupper på två personer arbetar med samma modul för att kunna diskutera eventuella problem och bolla idéer. Händer det att en person från gruppen är frånvarande ska den kvarvarande medlemmen kunna fortsätta med arbetet på egen hand tills den andra är tillbaka. För vissa större beslut ska en större del av projektgruppen konsulteras, annars ska arbetetet kunna drivas framåt parallellt mellan grupperna utan interaktion sinsemellan för att på så sätt maximera effektiviteten.

\section{Utbildningsplan}
Projektmedlemmarna får som förberedelse till projektet delta i sex stycken föreläsningar samt en labb i Mätteknik som ges i kursen Konstruktion med Mikrodatorer (TSEA29). Labben ska låta utbilda projektdeltagarna i att använda en logikanalysator. Kunden behöver ej genomgå någon särskild utbildning för att förstå projektets interna eller externa struktur.

\section{Rapporteringsplan}
En tidsrapport ska levereras till beställaren veckovis varje måndag senast kl 16:00 från och med 31/10 till och med 19/12. Leveransen ska ske från projektledaren och är projektledaren frånvarande ska första närvarande gruppmedlem enligt ordningen i tabellen på sida två i kravspecifikationen genomföra leveransen.

\section{Mötesplan}
De enda officiella mötena sker på måndagar i form av utvärderingsmöten (se avsnitt~\ref{subsec:villkorsamarbete}). Alla utvärderingsmöten antecknas och läggs upp på gruppens gemensamma mapp på Google Drive. Ifall ett extra möte krävs kan en gruppmedlem utlysa detta på Slack.

\section{Resursplan}
Detta avsnitt innehåller information om resurser inom projektet.
\subsection{Personer}
Gruppen består utav 6 personer som totalt ska arbeta 960 timmar med projektet efter beslutspunkt 2.
\begin{itemize}
  \item Patrik Sletmo
  \item Sebastian Callh
  \item Matilda Dahlström
  \item Anton Dalgren
  \item Rebecca Lindblom
  \item Matildha Sjöstedt
\end{itemize}
Alla i gruppen ska jobba lika mycket d.v.s. 160 timmar var efter beslutspunkt 2.
\newline\newline
Som resurs finns även en handledare att anlita vid behov. Om handledaren anser det nödvändigt hänvisar den till en teknisk expert inom det efterfrågade området.
\subsection{Material}
Projektet kräver flertalet komponenter och mätutrustning. All utrustning finns tillhandahållen av universitetet och behöver inte beställas. Projektet kommer kräva elektronikkomponenter, t.ex. processorer, sensorer och servon. Det kommer även att krävas en laptop med Bluetooth. Projektmedlemmarna behöver ha tillgång till varsin dator med tillhörande programvaror för utveckling.
\subsection{Lokaler}
Till projektets förfogande finns lokalerna Muxen 3-4 där dator och mätutrustning för hårdvaruutveckling finns att tillgå. Dessa lokaler är alltid tillgängliga efter beslutpunkt 2 och alla i gruppen kan vara där samtidigt. Vid eventuell platsbrist i Muxen kan gruppen hitta andra lediga ytor att arbeta på. Om det skulle ske så tas ett beslut inom gruppen om vilka medlemmar som behöver stanna kvar i Muxen.
\subsection{Ekonomi}
Projektet har en budget på max 160 arbetstimmar per person efter att beslutpunkt 2 har tagits.

\section{Milstolpar och beslutspunkter}
\subsection{Milstolpar}

\begin{center}
  \resizebox{\textwidth}{!}{\begin{tabular}{ | l | l | l | }
    \hline
    1 & Designspecifikation klar. & 2016-10-14 \\ \hline
    2 & Huvudbuss klar. & 2016-11-xx \\ \hline
    3 & Roboten rapporterar styr- och sensordata till huvudenheten. & 2016-11-xx  \\ \hline
    4 & Fungerande Bluetooth-kommunikation. & 2016-11-xx  \\ \hline
    5 & Roboten kan fjärrstyras från en dator. & 2016-11-xx \\ \hline
    6 & Roboten kan skicka data till mjukvaruklienten. & 2016-11-xx \\ \hline
    7 & Roboten kan fatta navigationsbeslut baserat på en karta. & 2016-11-xx \\ \hline
    8 & Huvudmodulen kan generera en karta. & 2016-11-xx \\ \hline
    9 & Mjukvaruklienten kan rendera kartan. & 2016-11-xx \\ \hline
    10& Roboten har korrekt styrreglering. & 2016-11-xx \\ \hline
  \end{tabular}}
\end{center}

\subsection{Beslutspunkter}
\begin{center}
  \resizebox{\textwidth}{!}{\begin{tabular}{ | l | l | l | }
    \hline
    3 & Godkännande av designspecifikationen, beslut att fortsätta uförandefasen & 2016-11-01 \\ \hline
    4 & Används ej &  \\ \hline
    5 & Godkännande av produktens funktionalitet, beslut att leverera & 2016-12-15 \\ \hline
    6 & Godkännande av leverans, beslut att upplösa projektgruppen & 2017-01-14 \\
    \hline
  \end{tabular}}
\end{center}

\section{Aktiviteter}
I tabellen nedan är aktiviteterna för projektet listade. Tiden som anges är en uppskattning och justering av tiden kan ske med hjälp av bufferttiden under projektets gång.
\begin{table}[H]
 \resizebox{\textwidth}{!}{\begin{tabular}{| l | l | l | l |}
 \hline
 \textbf{Nr} & \textbf{Aktiviet} & \textbf{Beror på} & \textbf{Tid (tim)} \\ \hline
 1 & Gör kopplingsschema för huvudenhet &   &  1 \\ \hline 
 2 & Gör kopplingsschema för sensorenhet &   & 1 \\ \hline
 3 & Gör kopplingsschema för styrenhet &  & 1 \\ \hline
 4 & Gör kopplingsschema för hela systemet & 1,2,3 & 1 \\ \hline
 5 & Undersök hur I2C implementeras & & 7 \\ \hline
 6 & Implementera masterrollen i I2C & 5 & 12 \\ \hline
 7 & Implementera slaverollen i I2C & 5 & 8 \\ \hline
 8 & Specifiera ett övergripande protokoll för huvudbussen & & 2 \\ \hline
 9 & Specifiera kommandon för kommunikation med styrenheten & & 4 \\ \hline
 10 & Specifiera kommandon för kommunikation med sensorenheten & & 3 \\ \hline
 11 & Undersök hur Bluetooth fungerar och implementeras & & 10 \\ \hline
 12 & Implementera Bluetooth i huvudmodulen & 11 & 8\\ \hline
 13 & Bestäm programspråk för mjukvaruklient & & 1 \\ \hline
 14 & Implementera Bluetooth i mjukvaruklient & 13 & 3 \\ \hline
 15 & Specifiera ett protokoll för kommunikation via Bluetooth & 11 & 3 \\ \hline
 16 & Specifiera kommandon för kommunikation med mjukvaruklienten & & 2 \\ \hline
 17 & Implementera protokollet för huvudbussen i master och slave & 8 & 10 \\ \hline
 18 & Implementera stub-funktioner för alla kommandon som  & 9 & 6 \\ 
    & styrenheten hanterar & & \\ \hline
 19 & Implementera stub-funktioner för alla kommandon som  & 10 & 6\\ 
    & sensorenhten hanterar & & \\ \hline
 20 & Implementera funktionalitet för att snurra hjulpar & 50 & 8 \\ \hline
 21 & Implementera funktionalitet för att styra laserservot & 50 & 8\\ \hline
 22 & Implementera funktionalitet för att ta emot data från hjulpar till & 7,20 & 5 \\
    & styrenheten och rapportera till huvudenheten & & \\ \hline
 23 & Implementera funktionalitet för att ta emot data från  & 7,21 & 5 \\
    & laserservot till styrenhet och rapportera till huvudeenhten & & \\ \hline
 24 & Läsa av data från laser & 50 & 10 \\ \hline
 25 & Läsa av data från accelerometer & 50 & 8 \\ \hline
 26 & Läsa av data från gyroskop 1 & 50 & 8 \\ \hline
 27 & Läsa av data från gyroskop 2 & 50 & 8 \\ \hline
 28 & Läsa av avbrott från knapp & 50 & 3\\ \hline
 29 & Implementera rapporteringsrutin för sensorenhet & 6,7 & 8\\ \hline
 30 & Koppla ihop I2C-buss & 4 & 2 \\ \hline
 31 & Designa navigeringsalgoritm & & 15 \\ \hline
 32 & Designa kartläggningsalgoritm & & 20 \\ \hline
 33 & Deisgna huvudlopp & & 10 \\ \hline
 34 & Koda navigeringsalgoritm & 31 & 10 \\ \hline
 35 & Koda kartläggningsalgoritm & 32 & 20 \\ \hline
 36 & Koda huvudloop & 33 & 10 \\ \hline
 37 & Deisnga UI & 13 & 4 \\ \hline
 38 & Implementera utritning av kartdata & 13,43 & 7\\ \hline
 39 & Implementera utrning av sensordata & 13,44 & 3\\ \hline
 40 & Implementera utritning av styrdata & 13,45 & 3 \\ \hline
 41 & Gör kontroller för fjärrstyrning av robot & 13 & 3 \\ \hline
 42 & Designa fallback-algoritm (ladda ner kartdata igen vid ev. krasch) & 12,14 & 8 \\ \hline
 43 & Se till att huvudenheten skickar kartdata till mjukvaruklienten & 12,14,35 & 4 \\ \hline
 44 & Se till att huvudenhten skickar sensordata till mjukvaruklienten & 12,14,29 & 4\\ \hline
 45 & Se till att huvudenhten skickar styrdata till mjukvaruklienten & 12,14,22 & 4 \\ \hline
 46 & Testa implementation & & 100 \\ \hline
 47 & Dokumentera & & 190 \\ \hline
 48 & Möten & & 120 \\ \hline
 49 & Buffert & & 233 \\ \hline
 50 & Montering och byggande & 4 & 30 \\ \hline
 
 
 \end{tabular}}
\end{table}
\section{Tidplan}
Tidplan för projektet bifogas som ett externt dokument.

\section{Förändringsplan}
Utifall att krav specificerade i kravspecifikationen ej kan uppnås eller tvingas omprioriteras, pga. till exempel tekniska problem eller tidsbrist, måste de nya kraven godkännas av beställaren. Vid större förseningar som resulterar i att robotens grundläggande krav inte kan uppfyllas innan projektets avslut (se Projektavslut) blir konsekvensen att projektet ej blir godkänt.

\section{Kvalitetsplan}
\subsection{Granskningar}
Dokumentansvarig är slutgiltiga ansvarig för att granska och godkänna dokument, även om samtliga gruppmedlemar ska se till att deras bidrag till dokumenten håller god kvalitet.

\subsection{Testplan}
\subsubsection{Enhetstester}
Allt eftersom kod skrivs ska det även skrivas enhetstester som verifierar den förväntade funktionaliteten. Innan ny eller förändrad funktionalitet flyttas till från en egen branch till master-branchen så ska den klara att köra igenom de enhetstester som finns skrivna.

\subsubsection{Systemtester}
När tillräckligt mycket funktionalitet implementerats för att utföra tester som påverkar systemet som stort kan systemtester utföras. Systemet sätts då i ett verkligt scenario där t.ex. kommunikation mellan modulerna eller navigationsimplementationen testas. Testarna kan utföras för att verifiera funktionalitet under utveckling men måste utföras för att verifiera milstolpar. I aktivitetsplanen finns det tid avsatt speciellt för att utföra dessa tester.

\section{Projektavslut}
Projektet avslutas genom en presentation (v 51, år 2016) inför beställare och kund samt en tävling enligt dokumentet \textit{Ban- och tävlingsspecifikation för kartrobotar 2016}, där andra robotar med samma funktionsmål deltar. Presentationen syfte är att t.ex. lyfta fram särskilda tekniska lösningar och ``göra reklam'' för vår grupp inför beställaren. All utlånad utrustning - inklusive robotens komponenter - ska återlämnas senast den 22/12 till handledaren. Projektet ska efter sitt slut utvärderas i en efterstudie. Här kommer ska bl.a. samarbetet i projektgruppen, relation med beställare och handledare samt tekniska problem diskuteras. Projektdeltagarna ska inte göra någon individuell uppföljning vad gäller sina färdigheter i projektarbete.

\end{document}
