\documentclass{article}
\usepackage[utf8]{inputenc}
\usepackage[english, swedish]{babel}

\usepackage{cite}
\usepackage{caption}
\usepackage{graphicx}
\usepackage{float}
\usepackage{textcomp}
\usepackage[yyyymmdd]{datetime}
\renewcommand{\dateseparator}{-}

\usepackage{graphicx}
\graphicspath{ {images/} }

%For headers & footers
\usepackage{fancyhdr}
\pagestyle{fancy}
\lhead{\includegraphics[scale=0.2]{Logo}}
\chead{Kartrobot}
\rhead{\today}

\lfoot{Konstruktion med mikrodatorer}
\rfoot{Grupp 3}

\renewcommand{\headrulewidth}{0.4pt}
\renewcommand{\footrulewidth}{0.4pt}


\title{Designspecifikation för kartrobot}
\author{Patrik Sletmo}
\date{\today}

\selectlanguage{swedish}

\begin{document}

\thispagestyle{empty}

{
\sffamily
\centering
\large


{\huge 
Designspecifikation för kartrobot
}

{\large
Patrik Sletmo
}

{\large
Version 0.1
}

\vspace{3.5cm}

Status
\begin{center}
\begin{tabular}{ | c | c | c | } 
\hline
\end{tabular}
\end{center}
}
\clearpage

\vspace*{\fill}
{
\sffamily
\centering
\large


{\huge
Projektidentitet
}

{\large
Grupp 3, 16/HT, KarToffel \\ Linköpings tekniska högskola, institution
}

\vspace{0.5cm}

\begin{table}[H]
\centering
\begin{tabular}{ | c | c | c | c |}
\hline
Namn & Ansvar & Telefon & E-post \\
\hline
Patrik Sletmo & Projektledare & 070 783 57 61 & patsl736@student.liu.se \\
\hline
Rebecca Lindblom &  & 073 436 40 79 & rebli156@student.liu.se \\
\hline
Matildha Sjöstedt &  & 070 515 84 11 & matsj696@student.liu.se \\
\hline
Sebastian Callh &  & 073 820 46 64 & sebca553@student.liu.se \\
\hline
Anton Dalgren &  & 076 836 51 56 & antda685@student.liu.se \\
\hline
Matilda Dahlström &  & 070 636 33 52 & matda715@student.liu.se \\
\hline
\end{tabular}
\end{table}
}

\begin{center}
\textbf{Hemsida}: https://github.com/SebastianCallh/kartoffel-tsea29
\end{center}

\begin{center}
\textbf{Kund}: Mattias Krysander, 013 - 28 2198 , matkr@isy.liu.se
\end{center}

\begin{center}
\textbf{Kursansvarig}: Tomas Svensson, 3B 528, +46 (0)13 28 1368, tomas.svensson@liu.se \\
\textbf{Handledare}: Anders Nilsson, 3B 512, +46 (0)13 28 2635, anders.p.nilsson@liu.se
\end{center}
\vspace*{\fill}
\clearpage

\renewcommand*\contentsname{Innehållsförteckning}
\tableofcontents
\clearpage


{
\sffamily
\centering
\large


{\huge 
Dokumenthistorik \\
}
\begin{center}
\begin{tabular}{ | c | c | c | c | c |} 
\hline
\textbf{Version} & \textbf{Datum} & \textbf{Utförda ändringar} & \textbf{Utförd av } & \textbf{Granskad} \\  
\hline
\end{tabular}
\end{center}
}

\clearpage


\section{Inledning}
% Kort beskrivning av hela dokumentet, skrivs förslagsvis sist

\clearpage

\section{Systembeskrivning}
% Beskrivning av systemets mest framträdande egenskaper

\subsection{Delsystem}
% Upplistning av delsystem, eventuellt någon kort kommentar om varje delsystem

\subsection{Övergripande konstruktion}
% Övergripande blockschema

\subsection{Komponenter}
% Väldigt kort beskrivning av delsektion

\subsubsection{Beräkningsenheter}
% Upplistning av processorer, SOC:s, etc

\subsubsection{Sensorer}
% Upplistning av alla sensorer

\subsubsection{Ställdon}
% Servon, motorer, etc

\subsubsection{Moduler}
% Upplistning av extrerna moduler, t.ex. Bluetooth (har vi ens några?)

\clearpage

\section{Delsystem}

\subsection{Huvudenhet}
% Kort beskrivning av huvudenhet

\subsubsection{Delsystemets funktion}
% Detaljerad beskrivning av delsystemets funktion

\subsubsection{Kopplingsschema}
% Helst ett kopplingsschema, alternativt väldigt detaljerat blockschema

\subsubsection{Komponenter}
% Upplistning av komponenter inkl. antal

\subsubsection{Resurser}
% Rada upp tillgängliga portar på mikroprocessorn samt hur många som krävs
% Motivera val av mikroprocessor med uppskattning av de resurser som krävs (prestanda, minne, IO)

\subsubsection{Programflöde}
% Hur ska mjukvaran i delsystemet fungera? Använd antingen psuedokod eller ett flödesschema

% Saxat från checklista:
%   Behövs det några speciella algoritmer för att lösa uppgiften?
%   Behövs det några större datastrukturer som kräver mycket minne?
%   Vilka avbrott ska användas och vad ska avbrottsrutinerna utföra?
%   Vilka funktioner ska utföras i en ”huvud-loop”?
%   Hur samverkar avbrottsrutinerna med huvudloopen?

\clearpage

\subsection{Sensorenhet}
% Se kommentarer för huvudenhet

\subsubsection{Delsystemets funktion}

\subsubsection{Kopplingsschema}

\subsubsection{Komponenter}

\subsubsection{Resurser}

\subsubsection{Programflöde}

\clearpage

\subsection{Styrenhet}
% Se kommentarer för huvudenhet

\subsubsection{Delsystemets funktion}

\subsubsection{Kopplingsschema}

\subsubsection{Komponenter}

\subsubsection{Resurser}

\subsubsection{Programflöde}

\clearpage

\subsection{Presentationsenhet}
% Se kommentarer för huvudenhet

\subsubsection{Delsystemets funktion}

\subsubsection{Blockschema}
% Övergripande schema för systemets stuktur

\subsubsection{Komponenter}

\subsubsection{Resurser}

\subsubsection{Programflöde}

\clearpage

\subsection{Fjärrstyrningsenhet}
% Se kommentarer för huvudenhet

\subsubsection{Delsystemets funktion}

\subsubsection{Blockschema}
% Övergripande schema för systemets stuktur

\subsubsection{Komponenter}

\subsubsection{Resurser}

\subsubsection{Programflöde}

\clearpage

\section{Kommunikation mellan delsystem}
% Kort inledning. Nämn att huvudbuss och Bluetooth använder ett "underliggande" protokoll (under I2C/Bluetooth) som specificeras under projektets gång

\subsection{Huvudbuss}
% Beskriv huvudbussen. Vad använder vi för teknik? Hur ser master/slave-förhållande ut?
% Använd med fördel \subsubsection

\subsection{Bluetooth}
% Beskriv Bluetooth. Vad använder vi för teknik? Hur ser master/slave-förhållande ut?
% Använd med fördel \subsubsection

\subsection{Informationsflöde}
% Vad skickas vart?
% Vi har sedan tidigare bestämt att allt skickas som "kommandon" vilket gör systemet helt asynkront,
% kommandon kan alltså inte direkt returnera information utan det sker vid ett senare tillfälle

\clearpage

\section{Implementationsstrategi}
% Eventuell kort inledning

\subsection{Utvecklingsstrategi}
% Beskriv hur vi utvecklar systemet till färdig produkt
% Inkludera t.ex. iterativ funktionalitet, minimal parallellism, risker (varför vi väljer att ha roboten tävlingsklar efter MS4), etc

\subsection{Testfilosofi}
% Hur har vi strukturerat upp arbetet efter testbarhet och hur är testtillfällen inplanerade?

\subsection{Feedback}
% Vad ger systemet oss för feedback?
% Värt att nämna aktivitet #3 samt senare skede då vi kan läsa av data i mjukvaruklienten

\subsection{Kritiska sektorer}
% Vad kan inte ändras när det väl är gjort eller blir så pass "dyrt" att göra om att det inte är möjligt?
% Försök hitta kritiska sektorer sent i projektets skede då de är "roligast"

\subsection{???}
% Står med i checklistan men vad betyder det: "Hur ofta bör man sampla?"

\clearpage

\nocite{*}
\bibliography{designspec}{}
\bibliographystyle{plain}

\end{document}
