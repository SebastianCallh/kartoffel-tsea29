\documentclass{article}
\usepackage[utf8]{inputenc}
\usepackage[english, swedish]{babel}

\usepackage{cite}
\usepackage{caption}
\usepackage{graphicx}
\usepackage{float}
\usepackage{textcomp}

\usepackage{listings}
\usepackage{color}
 
\definecolor{codegreen}{rgb}{0,0.6,0}
\definecolor{codegray}{rgb}{0.5,0.5,0.5}
\definecolor{codepurple}{rgb}{0.58,0,0.82}
\definecolor{backcolour}{rgb}{0.95,0.95,0.92}
 
\lstdefinestyle{mystyle}{
    backgroundcolor=\color{backcolour},   
    commentstyle=\color{codegreen},
    keywordstyle=\color{magenta},
    numberstyle=\tiny\color{codegray},
    stringstyle=\color{codepurple},
    basicstyle=\footnotesize,
    breakatwhitespace=false,         
    breaklines=true,                 
    captionpos=b,                    
    keepspaces=true,                 
    numbers=left,                    
    numbersep=5pt,                  
    showspaces=false,                
    showstringspaces=false,
    showtabs=false,                  
    tabsize=2
}

\lstset{style=mystyle}

\usepackage[yyyymmdd]{datetime}
\renewcommand{\dateseparator}{-}

\usepackage{graphicx}
\graphicspath{ {images/} }

%For headers & footers
\usepackage{fancyhdr}
\pagestyle{fancy}
\lhead{\includegraphics[scale=0.2]{Logo}}
\chead{Kartrobot}
\rhead{\today}

\lfoot{Konstruktion med mikrodatorer}
\rfoot{Grupp 3}

\usepackage{titlesec}

\setcounter{secnumdepth}{4}

\titleformat{\paragraph}
{\normalfont\normalsize\bfseries}{\theparagraph}{1em}{}
\titlespacing*{\paragraph}
{0pt}{3.25ex plus 1ex minus .2ex}{1.5ex plus .2ex}

\renewcommand{\headrulewidth}{0.4pt}
\renewcommand{\footrulewidth}{0.4pt}


\title{Efterstudie}
\author{Patrik Sletmo}
\date{\today}

\selectlanguage{swedish}

\begin{document}

\thispagestyle{empty}

{
\sffamily
\centering
\large


{\huge 
Efterstudie
}

{\large
Patrik Sletmo
}

{\large
Version 1.0
}

\vspace{3.5cm}

Status
\begin{table}[H]
\centering
\begin{tabular}{ | c | c | c | }
\hline
-- & Patrik Sletmo & 2016-12-XX \\
\hline
\end{tabular}
\end{table}
}
\clearpage

\vspace*{\fill}
{
\sffamily
\centering
\large


{\huge
Projektidentitet
}

{\large
Grupp 3, 16/HT, KarToffel \\ Linköpings tekniska högskola, ISY
}

\vspace{0.5cm}

\begin{table}[H]
\centering
\begin{tabular}{ | c | c | c | c |}
\hline
Namn & Ansvar & Telefon & E-post \\
\hline
Patrik Sletmo & Projektledare & 070 783 57 61 & patsl736@student.liu.se \\
\hline
Rebecca Lindblom & Utvecklare & 073 436 40 79 & rebli156@student.liu.se \\
\hline
Matildha Sjöstedt & Utvecklare & 070 515 84 11 & matsj696@student.liu.se \\
\hline
Sebastian Callh & Utvecklare & 073 820 46 64 & sebca553@student.liu.se \\
\hline
Anton Dalgren & Utvecklare & 076 836 51 56 & antda685@student.liu.se \\
\hline
Matilda Dahlström & Utvecklare & 070 636 33 52 & matda715@student.liu.se \\
\hline
\end{tabular}
\end{table}
}

\begin{center}
\textbf{Hemsida}: https://github.com/SebastianCallh/kartoffel-tsea29
\end{center}

\begin{center}
\textbf{Kund}: Mattias Krysander, 013 - 28 2198 , matkr@isy.liu.se
\end{center}

\begin{center}
\textbf{Kursansvarig}: Tomas Svensson, 3B 528, +46 (0)13 28 1368, tomas.svensson@liu.se \\
\textbf{Handledare}: Anders Nilsson, 3B 512, +46 (0)13 28 2635, anders.p.nilsson@liu.se
\end{center}
\vspace*{\fill}
\clearpage

\renewcommand*\contentsname{Innehållsförteckning}
\tableofcontents
\clearpage


{
\sffamily
\centering
\large


{\huge 
Dokumenthistorik \\
}
\begin{table}[H]
\centering
\begin{tabular}{ | c | c | c | c | c |} 
\hline
\textbf{Version} & \textbf{Datum} & \textbf{Utförda ändringar} & \textbf{Utförd av } & \textbf{Granskad} \\
\hline
1.0 & 2016-12-XX & Första version & Grupp 3 & Patrik Sletmo \\
\hline

\end{tabular}
\end{table}
}

\clearpage
\section{TIdsåtgång}
Text


\subsection{Arbetsfördelning}
Roboten består av en sedan tidigare byggd robotplattform kallad Terminator med ett antal virkort och komponenter monterade ovanpå. Eftersom roboten inte är byggd med konstruktionssäkerhet i åtanke bör försiktighet vidtagas när den ska förflyttas för att inte råka bryta någon av kopplingarna.


\subsection{Tidsåtgång jämfört med planerad tid}


\begin{table}[H]
\centering
\caption{En tabell över viktiga detaljer på roboten.}
\begin{tabular}{ | c | c | }
\hline
1 & Strömbrytare \\
\hline
2 & Batterianslutning \\
\hline
3 & Lasersensor \\
\hline
4 & Högra främre IR-sensor \\
\hline
5 & Vänstra främre IR-sensor \\
\hline
6 & Högra bakre IR-sensor \\
\hline
7 & Vänstra bakre IR-sensor \\
\hline
8 & Knapp för lägesändring \\
\hline
\end{tabular}
\label{table:components}
\end{table}
\ \\


\clearpage
\section{Anays av arbete och problem}
Text

\subsection{Vad hände under de olika faserna (bra/dåligt/orsak)?}
Text

\subsection{2.2 Hur vi arbetade tillsammans (ansvar, beslut, kommunikation etc.)?}
Text

\subsection{2.3 Hur använde vi projektmodellen?}
Text

\subsection{2.4 Hur fungerade relationen med beställaren?}
Text
\subsection{2.5 Hur fungerade relationen med handledaren?}
Text
\subsection{Tekniska framgångar /Prbolem}
Text


\clearpage
\section{Måluppfyllelese}

\subsection{4.1 De tre viktigaste erfarenheterna}

\subsection{4.2 Goda råd till de som ska utföra ett liknande projekt}


\nocite{*}
\bibliography{designspec}{}
\bibliographystyle{plain}

\end{document}
