\documentclass{article}
\usepackage[utf8]{inputenc}
\usepackage[english, swedish]{babel}

\usepackage{cite}
\usepackage{caption}
\usepackage{graphicx}
\usepackage{float}
\usepackage{textcomp}
\usepackage[yyyymmdd]{datetime}
\renewcommand{\dateseparator}{-}

\usepackage{graphicx}
\graphicspath{ {images/} }

%For headers & footers
\usepackage{fancyhdr}
\pagestyle{fancy}
\lhead{\includegraphics[scale=0.2]{Logo}}
\chead{Kartrobot}
\rhead{\today}

\lfoot{Konstruktion med mikrodatorer}
\rfoot{Grupp 3}

\renewcommand{\headrulewidth}{0.4pt}
\renewcommand{\footrulewidth}{0.4pt}


\title{Teknisk dokumentation för kartrobot}
\author{Patrik Sletmo}
\date{\today}

\selectlanguage{swedish}

\begin{document}

\thispagestyle{empty}

{
\sffamily
\centering
\large


{\huge 
Teknisk dokumentation för kartrobot
}

{\large
Patrik Sletmo
}

{\large
Version 1.0
}

\vspace{3.5cm}

Status
\begin{table}[H]
\centering
\begin{tabular}{ | c | c | c | }
\hline
STATUS & Patrik Sletmo & 2016-12-DD \\
\hline
\end{tabular}
\end{table}
}
\clearpage

\vspace*{\fill}
{
\sffamily
\centering
\large


{\huge
Projektidentitet
}

{\large
Grupp 3, 16/HT, KarToffel \\ Linköpings tekniska högskola, ISY
}

\vspace{0.5cm}

\begin{table}[H]
\centering
\begin{tabular}{ | c | c | c | c |}
\hline
Namn & Ansvar & Telefon & E-post \\
\hline
Patrik Sletmo & Projektledare & 070 783 57 61 & patsl736@student.liu.se \\
\hline
Rebecca Lindblom & Utvecklare & 073 436 40 79 & rebli156@student.liu.se \\
\hline
Matildha Sjöstedt & Utvecklare & 070 515 84 11 & matsj696@student.liu.se \\
\hline
Sebastian Callh & Utvecklare & 073 820 46 64 & sebca553@student.liu.se \\
\hline
Anton Dalgren & Utvecklare & 076 836 51 56 & antda685@student.liu.se \\
\hline
Matilda Dahlström & Utvecklare & 070 636 33 52 & matda715@student.liu.se \\
\hline
\end{tabular}
\end{table}
}

\begin{center}
\textbf{Hemsida}: https://github.com/SebastianCallh/kartoffel-tsea29
\end{center}

\begin{center}
\textbf{Kund}: Mattias Krysander, 013 - 28 2198 , matkr@isy.liu.se
\end{center}

\begin{center}
\textbf{Kursansvarig}: Tomas Svensson, 3B 528, +46 (0)13 28 1368, tomas.svensson@liu.se \\
\textbf{Handledare}: Anders Nilsson, 3B 512, +46 (0)13 28 2635, anders.p.nilsson@liu.se
\end{center}
\vspace*{\fill}
\clearpage

\renewcommand*\contentsname{Innehållsförteckning}
\tableofcontents
\clearpage


{
\sffamily
\centering
\large


{\huge 
Dokumenthistorik \\
}
\begin{table}[H]
\centering
\begin{tabular}{ | c | c | c | c | c |} 
\hline
\textbf{Version} & \textbf{Datum} & \textbf{Utförda ändringar} & \textbf{Utförd av } & \textbf{Granskad} \\
\hline
VERSION & 2016-12-DD & Första version & Grupp 3 & Patrik Sletmo \\
\hline

\end{tabular}
\end{table}
}

\clearpage

\section{Inledning}
% Bakgrund och syfte

Det här dokumentet skall ge en ingående beskrivning av kartrobot-systemet från ett tekniskt perspektiv.

\section{Produkten}
%En bild på produkten och en beskrivning av hur den fungerar. Beskriv vad den används till.

Systemet består av en fyrhjulig robot samt en mjukvaruklient som kommunicerar med varandra via Bluetooth. Systemets syfte är att roboten ska kunna kartlägga sin omgivning helt autonomnt och förmedla kartläggningen till mjukvaruklienten. För att göra det så är den utrustad med diverse IR-sensorer, en lasersensor och ett gyroskop för att kunna navigera och mäta avstånd. Se figur~\ref{fig:robot_bild} för en bild på konstruktionen.

\begin{figure}[H]
\centering
\includegraphics[scale=0.45]{Logo}
\caption{En bild på roboten.}
\label{fig:robot_bild}
\end{figure}

Mjukvaruklienten är som nämnt ansluten via Bluetooth, vilket låter den rendera kartläggningen i realtid. Den har även ett gränssnitt för att kunna fjärrstyra roboten. Se figur~\ref{fig:klient_bild} för en bild på mjukvaruklientens gränssnitt.

\begin{figure}[H]
\centering
\includegraphics[scale=0.45]{Logo}
\caption{En bild på mjukvaruklienten.}
\label{fig:klient_bild}
\end{figure}

\section{Teori}
% Beskrivning av regleralgoritmer mm.
Den övergripande strategin för att kratlägga rummet är att alltid färdas längst väggen på höger sida och på så sätt kartlägga rummets ytterväggar. För det krävs en regleringsalgoritm för att följa väggen och logik för navigering samt kartläggning.

\subsection{Navigering}
Navigeringslogiken implementeras som en tillståndsmaskin vars transitioner styrs av indata från robotens sensorer. Den antar att omgivningen följer banspecifikationen i bilaga~\ref{sec:banspec}.

Efter en kort uppvärmningsperiod för att hinna läsa in alla sensorer så börjar roboten följa väggen på sin högra sida och håller ett lämplig avstånd till den med hjälp av PD-reglering (se ~\ref{sec:reglering}). Om roboten upptäcker att väggen på höger sida tar slut så innebär det att väggen har svängt av bort från roboten och att den behöver göra en yttersväng till höger för att följa efter. Skulle roboten upptäcka en vägg framför sig betyder det istället att rummet svängt av till vänster, och att roboten behöver göra in innersväng åt vänster. Ett flödesdiagram över robotens navigeringstillstånd kan ses i figur~\ref{fig:navigator_flowchart}.

\subsection{Kartläggning}

\subsection{Reglering}
 ~\label{sec:reglering}
 
 \subsection{PWM}
~\label{sec:PWM}

\begin{figure}[H]
\centering
\includegraphics[scale=0.4]{navigator_flowchart}
\caption{Ett flödesdiagram på navigatorns tillstånd.}
\label{fig:navigator_flowchart}
\end{figure}


\section{Systemet}
% Ett översiktligt blockschema och en beskrivning av hela systemet.
TODO:Text om systemet i helhet

\begin{figure}[H]
\centering
\includegraphics[scale=0.4]{oversikt_systemet}
\caption{En bild på systemet övergripande konstruktion.}
\label{fig:oversikt_systemet}
\end{figure}

\section{Modulerna}

\subsection{Huvudenhet}
Huvudenheten realiseras med en Raspberry PI, på vilken operativsystemet Raspbian körs. Den representerar robotens beslutsfattande organ och sköter navigationsbeslut, kartläggning, kommunikation med mjukvaruklienten samt agerar mästare på robotens I2C-buss. På grund av dess många ansvarsområden är huvudmodulen uppdelad i undermodulerna kommunikationsenheten och logikenheten.

\subsubsection{Kopplingsschema}

\begin{figure}[H]
\centering
\includegraphics[scale=0.45]{Huvudenhet_kopplingsschema}
\caption{Kopplingsschema för huvudenheten.}
\label{fig:huvudenhet_kopplingsschema}
\end{figure}

Huvudenheten består av en Raspberry Pi 3 och en LCD-display av modellen JM162A. Utöver kopplingen till LCD:n så kopplas två pins till I2C-bussen. Eftersom Raspberry Pi drivs av 3.3 V matspänning och de övriga atmega-processorerna i roboten drivs av 5 V så behövs vi nivåskifta bussen så att de kan kommunicera med varandra, vilket görs med nivåskiftaren BSS138.

\subsubsection{Komponenter}

\begin{table}[H]
   \centering
  \begin{tabular}{ | c | c | }
    \hline
    \textbf{Komponent} & \textbf{Antal} \\
    \hline
    Raspberry PI 3 & 1 \\
    \hline
    JM162A & 1 \\
    \hline
  \end{tabular}
\end{table}

\subsubsection{Resurser}

\begin{table}[H]
   \centering
  \begin{tabular}{ | c | c | c | }
    \hline
    \textbf{Port} & \textbf{Antal} & \textbf{Används} \\
    \hline
    GPIO pins & 40 & 12 \\
    \hline
  \end{tabular}
\end{table}


\subsubsection{Logikenhet}
TODO:bild över alla pythonklasser tänker jag mig


\subsubsection{Kommunikationsenhet}
TODO:TEXT OM BLUETOOTH. UPPDATERA SÅ DEN REFLEKTERAR VÅR IMPLEMENTATION

\begin{figure}[H]
\centering
\includegraphics[scale=0.45]{Logo}
\caption{Flödesschema över Bluetooth-kommunikationen i huvudenheten.}
\label{fig:Kommunikation_huvudenhet_v2}
\end{figure}
\ \\



\subsection{Sensorenhet}
Sensorenheten har i uppgift att läsa in värden från robotens sensorer och rapportera värdena till huvudenheten. Den består av fyra IR-sensorer, ett gyroskop, en lasersensor och har sin egna beräkningsenhet, en ATMega 1284, vilket låter den arbeta asynkront från andra enheter. 

TODO:Pata fixar korrekt text här

Den ser på så sätt alltid till att ha data tillgänglig närhelst huvudenheten begär den. För IR-sensorerna sker det genom att processorn pollar information från varje sensor och sparar undan i minnet, medan gyroskop och laser håller koll på sina egna värden och är kopplade direkt på I2C-bussen. Huvudenheten kan sedan fråga sensorenheten efter ny IR-mätdata, eller fråga gyroskopet/lasern direkt varpå den får sensordatan levererad över I2C.

Avläsningarna från IR-sensorerna är analoga, och konverteras till digitalt med hjälp av ATMegans interna AD-omvandlare. För att sedan omvandla mätdatan till en approximation i millimeter används en tabell sparad i ATMegans minne.


\subsubsection{Kopplingsschema}
\begin{figure}[H]
\centering
\includegraphics[scale=0.6]{Sensorenhet_kopplingsschema}
\caption{Kopplingsschema för sensorenheten.}
\label{fig:sensorenhet_kopplingsschema}
\end{figure}

Som ses i figur~\ref{fig:sensorenhet_kopplingsschema} är IR-sensorerna kopplade till ATMegan via var sitt lågpassfilter för att reducera brus. Längst upp i kopplingsschemat ses den gemensamma I2C-bussen, dit gyroskop och laser är anslutna. Notera att EXO3-komponenten är delad mellan sensorenheten och styrenheten trots att den illustreras som separata komponenter i kopplingsschemat.

\subsubsection{Komponenter}

\begin{table}[H]
  \centering
  \begin{tabular}{ | c | c | c | c |}
    \hline
    \textbf{Komponent} & \textbf{Antal} \\
    \hline
    ATMega 1284 & 1 \\
    \hline
    LIDAR-Lite v2 & 1 \\
    \hline
    Knapp & 1 \\
    \hline
    GP2Y0A41SK IR-Sensor & 3 \\
    \hline
    GP2Y0A21 IR-Sensor & 1 \\
    \hline
    Adafruit 10-DOF IMU & 1 \\
    \hline
    EXO3 & 1 (delad) \\
    \hline
  \end{tabular}
  \caption{ Tabell över de komponenter som sensorenheten består av. }
\end{table}

\subsubsection{Resurser}
% Rada upp tillgängliga portar på mikroprocessorn samt hur många som krävs
% Motivera val av mikroprocessor med uppskattning av de resurser som krävs (prestanda, minne, IO)

TODO:TABELLEN NEDAN EJ KORRIGERAD
\begin{table}[H]
  \centering
  \begin{tabular}{ | c | c | c | c |}
    \hline
    \textbf{Port} & \textbf{Antal} & \textbf{Krävs} \\
    \hline
    SDA & 1 & 1 \\
    \hline
    SCL & 1 & 1 \\
    \hline
    PCINT & 24 & 1 \\
    \hline
    A/D & 8 & 2 \\
    \hline
    RESET & 1 & 1 \\
    \hline
    USART & 4 & 2 \\
    \hline
    JTAG & 1 & 1 \\
    \hline
    CLK & 1 & 1 \\
    \hline
  \end{tabular}
  \caption{Tabell över tillgängliga portar på processorn.}
\end{table}

Den maximala hastigheten som sensordata kan läsas in i är begränsad av sensorernas rapporteringsfrekvens som är betydligt långsammare än en ATmega 1284:s klockfrekvens, därför lönar det sig inte med en snabbare processor för sensorenhetens syfte. Det i princip enda minnet sensorenheten använder sig av är för temporärlagring av alla IR-sensorers data, vilket endast motsvarar fyra heltal. Utöver sensorinläsningen kommer enheten svara på kommandon över huvudbussen som inte heller kräver några höga hastigheter eller minneskrav.

\subsubsection{Programflöde}

\begin{figure}[H]
\centering
\includegraphics[scale=0.6]{sensorenhet_flowchart}
\caption{Ett flödesdiagram över sensorenhetens tillstånd}
\label{fig:sensorenhet_flowchart}
\end{figure}

För att kunna tillhandahålla sensordata från IR-sensorerna när huvudenheten ber den om det så läser sensorenheten in data med jämna mellanrum och sparar den lokalt för att sedan skicka datan på kommando från huvudenheten. Om den fått en begerän om sensordata från huvudenheten så tar det prioritet att leverera den över att läsa in ny data. Ett flödesdiagram över dess beteende ses i ~\ref{fig:sensorenhet_flowchart}.

\subsection{Styrenhet}
Styrenhetens uppgift är att kontrollera robotens förflyttning genom att ge korrekta signaler till dess hjulservon. Hjulen styrs parvis, det vill säga vänster hjulpar styrs med en signal och höger hjulpar med en annan. Hjulhastigheten styrs genom PWM (se avsnitt~\ref{sec:PWM})

TODO:Seba SKRIV KLART OCH STÄDA
 baserat på instruktioner ifrån huvudenheten. Den består av en ATMega 1284-processor som har robotens alla servon kopplade till sig och som tar emot instruktioner från huvudenheten över I2C-bussen. 

För att styra robotens hjul används PWM, där styrenheten är förprogrammerad med lämplig pulskvot.

\subsubsection{Kopplingsschema}
\begin{figure}[H]
\centering
\includegraphics[scale=0.45]{Styrenhet_kopplingsschema}
\caption{Kopplingsschema för styrenheten.}
\label{fig:styrenhet_kopplingsschema}
\end{figure}

TODO:TEXT OM KOPPLINGSSCHEMA

\subsubsection{Komponenter}

\begin{table}[H]
  \centering
  \begin{tabular}{ | c | c |}
    \hline
    \textbf{Komponent} & \textbf{Antal} \\
    \hline
    ATMega 1284 & 1 \\
    \hline
    Terminator (bas för fyrhjulingsrobot) & 1 \\
    \hline
    EXO3 & 1 (delad) \\
    \hline
    LS241 & 1 \\
    \hline
  \end{tabular}
\end{table}


\subsubsection{Resurser}
TODO:TABELL NEDAN EJ KORRIGERAD
\begin{table}[H]
  \centering
  \begin{tabular}{ | c | c | c | c |}
    \hline
    \textbf{Port} & \textbf{Antal} & \textbf{Krävs} \\
    \hline
    SDA & 1 & 1 \\
    \hline
    SCL & 1 & 1 \\
    \hline
    RESET & 1 & 1 \\
    \hline
    IO & 30 & 5 \\
    \hline
    USART & 4 & 4 \\
    \hline
    JTAG & 1 & 1 \\
    \hline
    CLK & 1 & 1 \\
    \hline
  \end{tabular}
  \caption{Tabell över tillgängliga portar på processorn.}
\end{table}

Omvandlingen av kommandon från huvudenheten och utskickning av signaler till servona uppskattas ej kräva mer prestanda eller minne än styrenhetens ATMega 1284 processor klarar av.

\subsubsection{Programflöde}

\begin{figure}[H]
\centering
\includegraphics[scale=0.6]{Styrenhet_flowchart}
\caption{Ett flödesschema över styrenhetens tillstånd}
\label{fig:styrenhet_flowchart}
\end{figure}

Styrenheten står hela tiden och väntar på instruktioner från huvudenheten, och så snart den får några utför den dem så snabbt den kan. ett flödesschema över styrenhetens beteende kan ses i figur ~\ref{fig:styrenhet_flowchart}.


\section{Slutsatser}
% Vilka förbättringar skulle kunna göras?

\subsection{Navigeringsalgoritm}
\label{sec:slutsatser_navigering}
Vår navigeringsalgoritm är i nuläget begränsad till att röra sig i ett rutnät och får således bara röra sig i vinklar som är multiplar av 90 grader. Begränsningen beror på att roboten inte har någon bra logik för att bestämma sin position utan att följa en vägg. Om roboten skulle kunna bestämma sin position när den navigerade i godtycklig riktning och på öppna ytor hade den totala sträckan som roboten behövt färdas minimeras.

\subsection{Kartläggning}
Roboten kartläggningsalgoritm kartlägger sin omgivning genom att beräkna raksträckor den åkt när den följt en vägg och spara start- och slut-koordinaterna. Om roboten kunde observera sin omgivning t.ex. med en svepande laser och beräkna var väggar fanns utifrån den datan så skulle den ej behöva färdas längs med alla väggar för att kartlägga dem. Det här förbättringsförslaget går hand i hand med förbättringarna föreslagna i avsnitt ~\ref{sec:slutsatser_navigering}.

\section{Referenser}
\appendix
%Appendix
%A. Kopplingsschema
%B. Programlistning och eventuell VHDL kod
%C. Övrigt (typ banspec)
\section{Banspecifikation}
\label{sec:banspec}

\end{document}
