\documentclass{article}
\usepackage[utf8]{inputenc}
\usepackage[english, swedish]{babel}

\usepackage{cite}
\usepackage{caption}
\usepackage{graphicx}
\usepackage{float}
\usepackage{textcomp}
\usepackage[yyyymmdd]{datetime}
\renewcommand{\dateseparator}{-}

\usepackage{graphicx}
\graphicspath{ {images/} }

%For headers & footers
\usepackage{fancyhdr}
\pagestyle{fancy}
\lhead{\includegraphics[scale=0.2]{Logo}}
\chead{Kartrobot}
\rhead{\today}

\lfoot{Konstruktion med mikrodatorer}
\rfoot{Grupp 3}

\renewcommand{\headrulewidth}{0.4pt}
\renewcommand{\footrulewidth}{0.4pt}


\title{Användarhandledning för kartrobot}
\author{Patrik Sletmo}
\date{\today}

\selectlanguage{swedish}

\begin{document}

\thispagestyle{empty}

{
\sffamily
\centering
\large


{\huge 
Användarhandledning för kartrobot
}

{\large
Patrik Sletmo
}

{\large
Version X
}

\vspace{3.5cm}

Status
\begin{table}[H]
\centering
\begin{tabular}{ | c | c | c | }
\hline
STATUS & Patrik Sletmo & 2016-12-DD \\
\hline
\end{tabular}
\end{table}
}
\clearpage

\section{Inledning}
Det här dokumentet innehåller all information som behövs för att kunna  använda kartroboten och dess mjukvaruklient. 

\section{Installation}
För att nyttja fjärrstyrningsfunktionaliteten hos roboten så behöver datorn man vill fjärrstyra ifrån ha robotens mjukvaruklient tillgänglig. SKRIV OM MJUKVARUKLIENTEN BEHÖVER INSTALLERAS/CONFAS NÅGOT HÄR

\section{Mjukvaruklienten}
Alla kommandon som roboten kan utföra kan skickas ifrån den tillhörande mjukvaruklienten.

\subsection{Ansluta till roboten}
SKRIV OM HUR MAN ANSLUTER MED MJUKVARUKLIENTEN HÄR 

\subsection{Kommunicera med roboten}
SKRIV OM MJUKVARUKLIENTEN MED EN BILD PÅ HUR 

\nocite{*}
\bibliography{designspec}{}
\bibliographystyle{plain}

\end{document}
